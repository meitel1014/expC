\documentclass[a4j,10pt,titlepage]{jsarticle}
\renewcommand{\headfont}{\bfseries}
\usepackage[dvipdfmx]{graphicx}
\title{情報科学演習C レポート2}
\author{藤田 勇樹 \\
大阪大学 基礎工学部 情報科学科 ソフトウェア科学コース\\
学籍番号: 09B16068 \\
メールアドレス: u461566g@ecs.osaka-u.ac.jp \\
担当教員\\
小島 英春 助教授 \\
内山 彰 助教授}
\date{提出日: 2018年5月26日}

\begin{document}
\maketitle
\section{課題2-1}
\subsection{概要}
この課題では,引数で与えられたホスト上で動作しているechoserverと通信し,標準入力から受け取った文字列をechoserverに送信し,echoserverから返された文字列をそのまま標準出力に表示するプログラムechoclientを作成する.echoserverとechoclientの通信にはTCPを用いる.

\subsection{仕様}
このプログラムの動作の流れは以下の通りである.
\begin{enumerate}
  \item 引数で与えられたホストで動作しているechoserverに接続する.
  \item 標準入力からの入力を受け付ける.
  \item 標準入力の内容がEOF(Ctrl-D)ならechoserverとの接続を切りプログラムを終了する.
  \item 標準入力の内容をechoserverに送信する.
  \item echoserverから返された文字列を表示する.
  \item 2.に戻る.
\end{enumerate}

以降の節では,1.の接続と4.および5.のデータの送受信について実装内容を説明する.

\subsection{接続の確立}
\subsubsection{ソケットの生成}
まず,通信の出入り口であるソケットを生成する.これには\verb|socket()|システムコールを用い,以下のように使用する.
\begin{verbatim}
sock=socket(AF_INET,SOCK_STREAM,IPPROTO_TCP)
\end{verbatim}
第一引数には通信方法を決定するプロトコルファミリーを指定する.ここではIPv4を使用するため\verb|AF_INET|を指定している.第二引数にはソケットの型を指定する.このプログラムではTCPを使用し,TCPは通常全二重バイトストリームのため\verb|SOCK_STREAM|を指定する.第三引数ではプロトコルを指定するため,\verb|IPPROTO_TCP|を指定している.

\subsubsection{接続先の設定}


\subsection{文字列の送受信}


\subsection{接続の切断}
接続を終了するには\verb|close()|システムコールを用いる.

\subsection{発展課題:lowerechoserver}
発展課題として,echoserverを改造し,echoclientから送られた文字列のうち大文字を小文字にして返すlowerechoserverを作成した.

具体的には,echoclientから受け取った文字列rbufに対し,大文字を小文字に変換する関数lowerを適用してからechoclientに送り返すようにした.

\section{課題2-2}
\subsection{発展課題:発言者の名前表示}
\end{document}
