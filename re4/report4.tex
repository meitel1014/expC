\documentclass[a4j,10pt,titlepage]{jsarticle}
\renewcommand{\headfont}{\bfseries}
\usepackage[dvipdfmx]{graphicx}
\title{情報科学演習C レポート4}
\author{藤田 勇樹 \\
大阪大学 基礎工学部 情報科学科 ソフトウェア科学コース\\
学籍番号: 09B16068 \\
メールアドレス: u461566g@ecs.osaka-u.ac.jp \\
担当教員\\
小島 英春 助教授 \\
内山 彰 助教授}
\date{提出日: 2018年8月3日}

\begin{document}
\maketitle
\section{チャットクライアント}
クライアントの実装について述べる.基本的にプロトコルの状態遷移に沿った説明を行うが,実際のプログラムでは逐次処理やループで実装できたため,状態変数を用意して分岐するといった厳密な状態遷移は行っていない.サーバの実装も同様.

\subsection{状態c1 ---初期状態---}
ここでは,サーバへの接続を行う.\verb|socket()|でソケットを生成し,接続先情報を設定し,\verb|connect()|で接続を行う.

\subsection{状態c2 ---参加---}
\verb|read()|でサーバからソケットを通じて17バイトだけメッセージを受信する\footnotemark.この内容が\verb|"REQUEST ACCEPTED\n"|であれば,接続が受理されたことになる.それ以外であった場合,接続が受理されていないためエラー処理であるc6へ移行する.

\footnotetext{資料のプロトコルでは文字列の終端のNULL文字を必須としていないため,NULL文字は付与されていないことを前提にしている.代わりに,今回作成したプログラムでは基本的に改行文字を文字列の終端に付与するようにした.\verb|<string.h>|のライブラリ関数を使用する場合は,一部を除いて,受信したメッセージの最後のバイトをNULL文字に置き換えてC文字列としている.}

\subsection{状態c3 ---ユーザ名登録---}
引数で指定されたユーザ名に改行を付与し,ソケットに\verb|write()|する.その後,\verb|read()|で20バイト受信し,内容が\verb|"USERNAME REGISTERED\n"|であれば,ユーザ名が受理されたことになる.そうでない場合,c6へ移行する.

\subsection{状態c4 ---メッセージ送受信---}
システムコール\verb|select()|を用いて,標準入力とソケットからの入力を同時に監視し,標準入力からの入力はソケットに\verb|write()|してサーバに送信,ソケットからの入力は標準出力に出力する.なお,標準入力からの入力が\verb|EOF|(Ctrl-D)であった場合は代わりに状態c5に移行する.それ以外の場合はc4を繰り返す.

\subsection{状態c5 ---離脱---}
\verb|close()|でソケットを閉じてサーバとの通信を終了し,プログラムを終了する.

\subsection{状態c6 ---例外処理---}
\verb|close()|でソケットを閉じてサーバとの通信を終了し,プログラムを異常終了する.

\subsection{拡張機能}
\subsubsection{プロンプト表示}\label{sec:prompt}
状態c4でキーボードからの入力を受け付ける際,プロンプトとして``(ユーザ名) \verb|>|''を表示するようにした.これにより,自分の入力が分かりやすくなる.ただし,これだけだと自分の送ったメッセージがサーバから返される際に同じ表示が2つ続いてしまうので,プロンプトと入力した内容は送信時に端末上の表示を消去するようにした.ただし,\ref{sec:extsrv}節で述べる\verb|/list|や\verb|/send|から始まるメッセージはそのままサーバから返ってくることはないので,その場合に限り端末上の表示は消去しない.

\section{チャットサーバ}
\subsection{ソケットとユーザ名の管理}
ソケットとユーザ名の管理にはリスト構造を用いた.構造体socklistを以下で定義する.
\begin{verbatim}
struct socklist {
    int csock;
    char username[101];
    struct socklist* next;
}
\end{verbatim}

csockにはクライアントと通信するソケットのファイルディスクリプタを,usernameにはユーザ名を,nextには次のsocklistへのポインタを格納する.

また,これとは別にグローバル変数sockheadとcsocknumを定義し,それぞれリストの先頭とクライアント数kを表す変数とした.

\subsection{状態s1 ---初期状態---}
\verb|socket()|でソケットを生成し,接続先情報を設定し,\verb|bind()|と\verb|listen()|で接続要求を待機するようにする.

\subsection{状態s2 ---入力待ち---}
接続要求を待機するソケットとクライアントと通信するソケットすべてからの入力を\verb|select()|で同時に監視する.監視するファイルディスクリプタ集合\verb|rfds|はクライアントが参加/離脱する度に更新する必要があることに注意する.

\subsection{状態s3 ---入力処理---}
\verb|select()|で監視した結果,接続要求を待機するソケットからの入力であれば状態s4へ,クライアントからの入力であれば状態s6へ移行する.

その後,状態s2へ移行する.

\subsection{状態s4 ---参加受付---}
\verb|accept()|でソケットからの接続要求を受理する.csocknumがMAXCLIENTS以下なら接続を受理したことを表す\verb|"REQUEST ACCEPTED\n"|を受け付けたクライアントに送信し,クライアントリストに追加(newsockとする),csocknumを1増やし,状態s5へ移行する.csocknumがMAXCLIENTSを超えていた場合は,接続を拒否したことを表す\verb|"REQUEST REJECTED\n"|を送信し,受け付けたソケットを閉じ,状態s3へ移行する.

\subsection{状態s5 ---ユーザ名登録---}
次にユーザ名+改行の形でユーザ名が送信されるため,\verb|read()|で受信し,クライアントリストに同じユーザ名のものがあれば登録拒否とし,\verb|"USERNAME REJECTED\n"|を送信してソケットを閉じ,newsockをクライアントリストから削除し,csocknumを1減らして状態s3に移行する.同じユーザ名がなければ,newsockのusernameを受信したユーザ名にし,\verb|"USERNAME REGISTERED\n"|を送信する.その後状態s3に移行する.

\subsection{状態s6 ---メッセージ配信---}
\verb|select()|が反応したクライアントソケットに対し,\verb|"read()"|でメッセージを受信する.内容がEOF(\verb|read()|の返り値が0)なら状態s7に移行する.それ以外の場合,受信したメッセージの先頭にユーザ名を追加し\verb|"username >message"|の形にして,クライアントリストの全てのソケットに\verb|"write()"|する.その後状態s3に移行する.

\subsection{状態s7 ---離脱処理---}
EOFを送信したクライアントのソケットを\verb|"close()"|し.クライアントリストから該当する要素を削除,csocknumを1減らし,状態s3に移行する.

\subsection{拡張機能}\label{sec:extsrv}
\subsubsection{ユーザ名リスト}
状態s6において,受信したメッセージが\verb|"/list"|で始まっていれば,そのメッセージを送信したクライアントにその時点で接続しているユーザの一覧を送信する.

\subsubsection{秘密のメッセージ}
状態s6において,受信したメッセージが\verb|"/send username message"|の形であれば,ユーザ名がusernameのクライアントにのみ,messageの内容を送信する.このとき,送信するメッセージは,\verb|"(送信元のユーザ名)*>message"|とし,通常のメッセージと区別する.ユーザ名がusernameであるクライアントが接続していなかった場合や,\verb|"/send"|から始まっているが形式が不適切な場合,送信元に使用方法を送信する.

\subsubsection{参加,離脱の通知}
クライアントが参加/離脱した時,全クライアントに\verb|"Server >(user) joined/exited:(users) online"|を送信する.(user)には参加/離脱したクライアントのユーザ名が,(users)には参加中のクライアント数が入る.

\subsubsection{サーバ終了の予告}
サーバにCtrl-Cを入力しプログラムを終了しようとすると,その時点で参加しているクライアントにあと10秒で終了する旨を送信し,その10秒後にプログラムを終了する.この間に再びCtrl-Cが入力された場合,この処理を最初からやり直す(最後にCtrl-Cを入力してから10秒で終了する).

\section{実行結果}
以下,exp046でサーバを,exp051とexp064でクライアントを実行する.

クライアント数がMAXCLIENTSを超える時の動作を確認する.chatserverのMAXCLIENTSを1に設定した.
\begin{verbatim}

\end{verbatim}



\ref{sec:prompt}

\begin{verbatim}
yk-fujit@exp051:~/expC/re4$ ./chatclient exp046 clt
REQUEST ACCEPTED
USERNAME REGISTERED
Server >clt joined:1 online
Server >sample joined:2 online
clt >Hello
sample >hi
sample >aiueo
clt >abcde
clt >bye
closed
yk-fujit@exp051:~/expC/re4$ ./chatclient exp046 clt
REQUEST ACCEPTED
USERNAME REGISTERED
Server >clt joined:2 online
Server >10 seconds remaining
clt >oh
closed
yk-fujit@exp051:~/expC/re4$ 

yk-fujit@exp064:~/expC/re4$ ./sampleclient exp046 sample
ICS Exercises C sample program chatclient.c
connected to exp046
join request accepted
user name registered
Server >sample joined:2 online
clt >Hello
hi
sample >hi
aiueo
sample >aiueo
clt >abcde
clt >bye
Server >clt exited:1 online
Server >clt joined:2 online
Server >10 seconds remaining
clt >oh
yk-fujit@exp064:~/expC/re4$ 

yk-fujit@exp046:~/expC/re4$ ./chatserver 
NEW USER:clt
NEW USER:sample
client clt closed
NEW USER:clt
Server stopping...
yk-fujit@exp046:~/expC/re4$ 
\end{verbatim}

\begin{verbatim}
yk-fujit@exp051:~/expC/re4$ ./chatclient exp046 clt
REQUEST ACCEPTED
USERNAME REGISTERED
clt >hello
sample >test
sample >aaa
clt >doumo
clt >bye
closed
yk-fujit@exp051:~/expC/re4$ 

yk-fujit@exp064:~/expC/re4$ ./sampleclient exp046 sample
ICS Exercises C sample program chatclient.c
connected to exp046
join request accepted
user name registered
clt >hello
test
sample >test
aaa
sample >aaa
clt >doumo
clt >bye
yk-fujit@exp064:~/expC/re4$

yk-fujit@exp046:~/expC/re4$ ./sampleserver 
ICS Exercises C sample program chatserver
user<clt> connected
user<sample> connected
user<clt> closed
user<sample> closed
\end{verbatim}

\begin{verbatim}
yk-fujit@exp051:~/expC/re4$ ./chatclient exp046 clt
REQUEST ACCEPTED
USERNAME REGISTERED
Server >clt joined:1 online
clt >/list
Server >user list:
clt
Server >sample joined:2 online
clt >/list
Server >user list:
sample
clt
clt >/send sample secret
sample*>i got it
clt >/send aaa dummy
Server >aaa:No such user online
clt >/send
Server >Usage:/send username message
clt >/send sample secret message
Server >sample exited:1 online
clt >/send sample secret message
Server >sample:No such user online
clt >/list
Server >user list:
clt
Server >sampleaaa joined:2 online
clt >/list
Server >user list:
sampleaaa
clt
clt >

yk-fujit@exp064:~/expC/re4$ ./sampleclient exp046 sample
ICS Exercises C sample program chatclient.c
connected to exp046
join request accepted
user name registered
Server >sample joined:2 online
clt*>secret
/send clt i got it
yk-fujit@exp064:~/expC/re4$ ./sampleclient exp046 sampleaaa
ICS Exercises C sample program chatclient.c
connected to exp046
join request accepted
user name registered
Server >sampleaaa joined:2 online

yk-fujit@exp046:~/expC/re4$ ./chatserver 
NEW USER:clt
NEW USER:sample
client sample closed
NEW USER:sampleaaa
\end{verbatim}

\begin{verbatim}
yk-fujit@exp051:~/expC/re4$ ./chatclient exp046 clt
REQUEST ACCEPTED
USERNAME REGISTERED
Server >clt joined:1 online

yk-fujit@exp064:~/expC/re4$ ./chatclient exp046 clt
REQUEST REJECTED
yk-fujit@exp064:~/expC/re4$ /home/exp/staff/utiyama/public/enshu-c_sample/chatclient exp046 sample
ICS Exercises C sample program chatclient.c
connected to exp046
join request rejected

yk-fujit@exp046:~/expC/re4$ ./chatserver 
NEW USER:clt
REQUEST REJECTED:Server full
REQUEST REJECTED:Server full
\end{verbatim}

\section{考察}

\section{感想}

\end{document}
