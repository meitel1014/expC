\documentclass[a4j,10pt,titlepage]{jsarticle}
\renewcommand{\headfont}{\bfseries}
\usepackage[dvipdfmx]{graphicx}
\title{情報科学演習C レポート3}
\author{藤田 勇樹 \\
大阪大学 基礎工学部 情報科学科 ソフトウェア科学コース\\
学籍番号: 09B16068 \\
メールアドレス: u461566g@ecs.osaka-u.ac.jp \\
担当教員\\
小島 英春 助教授 \\
内山 彰 助教授}
\date{提出日: 2018年7月2日}

\begin{document}
\maketitle
\section{課題3-1}
\subsection{概要}
この課題では,\verb|fork()|した4つのプロセスで順にファイルを処理するつもりが,競合状態に陥ってしまうプログラムfile-counterを正しく動作するように修正する.

\subsection{改造内容}
元々のプログラムでは,まず``0''と書き込んだファイルcounterを作成した後,プロセスを4回forkし,それぞれの子プロセスでcounterファイルを開いてそれに書いてある数字を読み取り,1を足して書き込む,という操作を行う.この操作によりプログラム終了時のcounterファイルの内容は``4''になることが想定されている.しかし,元々のプログラムのままでは,プロセス間の同期が取れていないために,あるプロセスがcounterファイルを読み込んで書き込むまでに,違うプロセスがcounterファイルを読み取ってしまうことがある.その結果,プログラム終了時のcounterファイルの内容は,実際には2や3になっている.

これを修正し排他制御を実現するために,セマフォという同期システムを用いる.セマフォには0以上の整数を保存し,これに対しwaitとsignalという操作が可能である.waitをすると,セマフォの値が1以上であれば減らし,0ならば実行待ちキューにwait操作をしたプロセスをプッシュする.一方signalを行うと,実行待ちキューにプロセスがあればプロセスを一つポップしてそのプロセスの実行を再開し,なければセマフォの値を増やす.

file-counter中でのセマフォの使用方法について述べる.各子プロセスの処理内容は以下の通りである.
\begin{enumerate}
  \item wait
  \item counterファイルを読み取り1加えた値を書き込む(count1)
  \item signal
  \item 子プロセス終了
\end{enumerate}

セマフォの初期値は1に設定してあるため,子プロセスを4つforkした後,最初に実行される子プロセスはwaitでセマフォの値を0にして処理を継続する.その他の子プロセスがwaitする時にはセマフォの値は0になっているから,処理を一時停止し実行待ちキューに格納される.最初の子プロセスのcount1の実行が終わりsignalを行うと,実行待ちキュー内の子プロセスが一つ再開される.このとき最初の子プロセスのcount1はすでに終了しているため,ファイルcounterの値は1になっている.これを繰り返すと,それぞれの子プロセスは前の子プロセスのcount1の終了を待って自分のcount1を行うことになるため,counterファイルの最終的な値は4になる.

\subsection{実行結果}
\begin{verbatim}
$ ./file-counter
count = 1
count = 2
count = 3
count = 4
$ ./file-counter
count = 1
count = 2
count = 3
count = 4
$ ./file-counter
count = 1
count = 2
count = 3
count = 4
\end{verbatim}

プロセス間の同期がとれたことで,確実にcountの値が1ずつ増え,何度実行しても結果が変わらなくなった.

\section{課題3-2-1}
\subsection{概要}
この課題では,パイプで子プロセスから親プロセスへ文字列を送信するプログラムを拡張し,逆に親プロセスから子プロセスへも同時に送信するプログラムtwo-way-pipe.cを作成する.

パイプとは,プロセス間の通信に用いるために使用する領域で,ファイルディスクリプタを通じてアクセスするため一種のファイルのように扱うことができる.パイプは\verb|pipe()|システムコールで生成し,\verb|read()|や\verb|write()|を使用して読み書きが可能である.

\subsection{改造内容}
元々のプログラムでは,子プロセスから親プロセスへのパイプのみ生成されているため,同様にして親プロセスから子プロセスへのパイプを作成する.また,子プロセスではパイプへのwrite,親プロセスではパイプからのreadを行っているが,それぞれを両方のプロセスで行うようにする.

\subsection{実行結果}
\begin{verbatim}
$ ./two-way-pipe hello HELLO
message from child process: 
	hello
message from parent process: 
	HELLO
\end{verbatim}

\section{課題3-2-2}
\subsection{概要}
この課題では,マージソートのプログラムを拡張し,2プロセスで並列に手分けしてソートし,それぞれの結果をパイプを用いて一つのプロセスにまとめ,最後にそれらをマージするプログラムを作成する.

\subsection{改造方法}
まず,マージソートの最初のステップでは配列を2分割するが,このうち一つを親プロセスで,もう一つを子プロセスでそれぞれマージソートするようにする.これが終了すると通常のマージソートにおける最後のマージ直前の状態になる.その後,子プロセスでマージソートした後の配列をパイプを経由して親プロセスに送信し,親プロセスが自分でマージソートした配列とパイプで受信した配列をマージすれば,マージソートが完了する.

\begin{verbatim}
$ ./pipemerge 
Done with sort.
52892392
262577817
627763308
700818707
868170278
984698191
1126456746
1162516543
1607248803
1692208745
\end{verbatim}
正しくソートできている.

\section{課題3-3-1}
\subsection{概要}
この課題では,システムコール\verb|alarm()|の機能を実現する関数myalarmを作成する.

\subsection{実装方法}
myalarm内で\verb|fork()|を行い,親プロセスはそのまま呼び出し元に戻り,子プロセスは指定された秒数だけ\verb|sleep()|した後親プロセスに\verb|SIGALRM|を送ればよい.

ただし,単純にこれを行うだけではいくつかの問題がある.まず,要求仕様上,myalarmが\verb|SIGALRM|を送る前にmyalarmを再度呼び出すとmyalarmのタイマーをリセットしなければならない.これを実現するために,前回のmyalarm呼び出し時に作成した子プロセスのプロセスIDをstatic変数で記憶しておき,再度呼び出された際は前回の子プロセスに\verb|SIGKILL|を送り中断させる.

もう一つの問題として,通常\verb|fork()|で作成した子プロセスが使用するリソースは子プロセスの終了後親プロセスが\verb|wait()|することで解放されるが,myalrmを呼び出して戻ってきた親プロセスは続きの処理を行うため,子プロセスを\verb|wait()|することができない.そのため子プロセスは\verb|SIGALRM|を送った後ゾンビ状態となってしまう.これを解決するために,\verb|sigaction()|システムコールを用いる.これは\verb|signal()|システムコールのようにシグナルへの応答を設定するものであるが,より詳細な動作を設定することができる.ここでは,シグナルハンドラとしてそのシグナルを無視する(何もしない)\verb|SIG_IGN|を設定し,さらに,追加の動作として,設定したシグナルが\verb|SIGCHLD|の場合,子プロセスが終了したときにゾンビ状態にしないようにする\verb|SA_NOCLDWAIT|を設定する.これにより親プロセスが\verb|wait()|しなくても子プロセスは終了後ゾンビ状態にならない.

\subsection{実行結果}
\begin{verbatim}
$ ./alarm
(10秒経過)
This program is timeout.
(プログラム終了)
$ ./alarm
a
echo: a
(aを入力してから10秒経過)
This program is timeout.
(プログラム終了)
$
\end{verbatim}
アラームが正しく動作し,アラームのリセットもできている.
また,psコマンドで確認するとタイムアウト後子プロセスもゾンビプロセスにならずに終了している.

\section{課題3-3-2}
\subsection{概要}
この課題では,前節で作成したmyalarmを,前回の課題で作成したsimple-talk-clientに組み込み,一定時間発言しなかった場合にタイムアウトする機能を作成する.

\subsection{改造内容}
simple-talk-clientでは標準入力とソケットを同時に監視するため\verb|select()|システムコールを用いているが,これの実行中にシグナルハンドラの割り込みが発生すると,\verb|select()|はエラー扱いとなり,-1を返して\verb|errno|を\verb|EINTR|に設定する.よってこの状態になった時,ソケットを\verb|close()|してプログラムを終了すればよい.

\subsection{実行結果}
前回の拡張課題として実装した,発言者の名前を表示する機能はそのままにしてある.
\begin{verbatim}
$ ./simple-talk-client exp101
Enter your name>CLT
connected with SVR
CLT > hello
(hello入力後10秒経過)
connection timed out.
closed
(プログラム終了)
$
\end{verbatim}

\section{発展課題1}
\subsection{概要}
この課題では,セマフォを利用して,すべてのプロセスが処理中のある時点までたどり着くまで他のプロセスは待機するバリア同期を実装する.

\subsection{仕様}
このプログラムはまず,引数で与えられた数だけ\verb|fork()|し子プロセスを作成する.各子プロセスは,自身のプロセスIDを5で割った余りの秒数だけ\verb|sleep()|し,その後``Child process (pid) ended''を出力し終了する.親プロセスはすべての子プロセスが終了したのを確認してから,``All child processes ended''を出力して終了する.

\subsection{実装方法}
各子プロセスは終了時にsignalを行い,親プロセスはsignalを子プロセスの数だけwaitするようにする.これにより,親プロセスはすべての子プロセスの終了を待つことができる.これはwait用のsembufのsem\_opを-1から-n(nは子プロセスの数)にすれば実現できる.

\subsection{実行結果}
\begin{verbatim}
$ ./barrier 10
Child process 14975 ended
Child process 14980 ended
Child process 14976 ended
Child process 14971 ended
Child process 14977 ended
Child process 14972 ended
Child process 14978 ended
Child process 14973 ended
Child process 14979 ended
Child process 14974 ended
All child processes ended
\end{verbatim}
すべての子プロセスの終了の出力の後に親プロセスが出力を行っている.

\section{発展課題3}
\subsection{概要}
この課題では,ここまでの課題で学んだセマフォやシグナルを用いて,2プロセスが交互に動作するようなプログラムを作成する.

\subsection{仕様}
プログラムを起動すると,\verb|fork()|したプロセスAとBが,あらかじめランダムに作成した配列AとBの内容を交互に表示する.プロセスAは配列Aを,プロセスBは配列Bを表示する.

\subsection{実装方法}
セマフォが1つだと,次にどちらのプロセスが実行されるか分からないため,セマフォを2つ用いた.具体的には,プロセスAでは,

\begin{enumerate}
  \item セマフォ1をwait
  \item Aの内容を一つ出力
  \item セマフォ2にsignal
\end{enumerate}

の処理を繰り返し,プロセスBでは,

\begin{enumerate}
  \item セマフォ2をwait
  \item Bの内容を一つ出力
  \item セマフォ1にsignal
\end{enumerate}

を繰り返すようにした.ここで,セマフォ1の初期値を1に,セマフォ2の初期値を0にすると,

\begin{enumerate}
  \item プロセスAがwaitから復帰,処理を行う(セマフォ1=0,セマフォ2=0)
  \item プロセスAがsignal(セマフォ1=0,セマフォ2=1)
  \item プロセスBがwaitから復帰,処理を行う(セマフォ1=0,セマフォ2=0)
  \item プロセスBがsignal(セマフォ1=1,セマフォ2=0)
\end{enumerate}

これを繰り返すことになり,仕様通りの動作が行える.

\subsection{実行結果}
\begin{verbatim}
$ ./rotation 
A[0] = 2899
B[0] = 6064
A[1] = 4806
B[1] = 5391
A[2] = 8125
B[2] = 9354
A[3] = 3773
B[3] = 6435
A[4] = 8643
B[4] = 1452
(中略)
B[96] = 2722
A[97] = 5798
B[97] = 8942
A[98] = 5284
B[98] = 6991
A[99] = 1253
B[99] = 991
\end{verbatim}

正しく交互に出力できている.

\section{感想}
今回扱ったパイプや前回扱ったソケット等はファイルディスクリプタを通じて一種のファイルとして扱えるため,統一感があり非常に使いやすいものだと感じた.

\end{document}
